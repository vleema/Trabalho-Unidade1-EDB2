\section{Implementação dos algoritmos de ordenação}
\label{sec:impl-algo-ord}

Os algoritmos de ordenação \textbf{Merge Sort}, \textbf{Quick Sort} e \textbf{Bubble Sort} foram implementados em suas versões iterativas e recursivas na linguagem de programação Rust. Além disso, com a exceção do \textbf{Bubble sort}, todas as funções esperam um slice mutável de tipo \texttt{T} que implementa uma ordem parcial e pode ser copiado, estes são os traits \texttt{PartialOrd} e \texttt{Copy}, respectivamente. Todas as funções alteram esse slice diretamente e não retornam nada.

\subsection{Bubble Sort}

\subsubsection{Iterativo}

\begin{lstlisting}[language=Rust]
pub fn iterative_bubble_sort<T: PartialOrd>(arr: &mut [T]) {
  for i in 0..arr.len() {
    for j in 1..arr.len() - i {
      if arr[j - 1] > arr[j] {
        arr.swap(j - 1, j);
      }
    }
  }
}
\end{lstlisting}

\subsubsection{Recursivo}

Na versão recursiva, foi utilizada a função auxiliar \texttt{bubble\_sort\_pass}

\begin{lstlisting}[language=Rust]
pub fn recursive_bubble_sort<T: PartialOrd>(arr: &mut [T]) {
  if arr.is_empty() {
    return;
  }
  let last_element_position = arr.len();
  bubble_sort_pass(arr, 1, last_element_position);
  recursive_bubble_sort(&mut arr[..last_element_position - 1]);
}
\end{lstlisting}

\subsubsection{Função auxiliar}

Essa função move o maior elemento de um slice para a posição \texttt{last\_element\_position} no \texttt{arr}. O \texttt{iterator} é usado para percorrer o slice do inicio ao fim.

\begin{lstlisting}[language=Rust]
fn bubble_sort_pass<T: PartialOrd>(arr: &mut [T], iterator: usize, last_element_position: usize) {
  if iterator >= last_element_position {
    return;
  }
  if arr[iterator - 1] > arr[iterator] {
    arr.swap(iterator - 1, iterator)
  }
  bubble_sort_pass(arr, iterator + 1, last_element_position)
}
\end{lstlisting}
\FloatBarrier

\subsection{Quick Sort}

\subsubsection{Iterativo}

Como o \textit{quick sort} naturalmente é um algoritmo recursivo, precisamos de algo que simulasse a pilha de execução do computador, esta simulação se deu a partir do uso da variável \texttt{stack} que armazena as posições das sub-listas que precisam ser ordenadas, permitindo a progressão iterativa da divisão da lógica de divisão e conquista.

\begin{lstlisting}[language=Rust]
pub fn iterative_quick_sort<T: PartialOrd + Copy>(arr: &mut [T]) {
  if arr.is_empty() {
    return;
  }
  let mut stack = vec![(0, arr.len() - 1)];
  while let Some((low, high)) = stack.pop() {
    if low < high {
      let pivot_index = partition(arr, low, high);
      if pivot_index > 0 {
        stack.push((low, pivot_index - 1)); // Left side
      }
      stack.push((pivot_index + 1, high)); // Right side
    }
  }
}
\end{lstlisting}
\FloatBarrier

\subsubsection{Recursivo}

Para manter o único parâmetro \texttt{arr}, a \texttt{recursive\_quick\_sort} só encapsula a versão de fato recursiva.

\begin{lstlisting}[language=Rust]
pub fn recursive_quick_sort<T: PartialOrd + Copy>(arr: &mut [T]) {
  if arr.is_empty() {
    return;
  }
  _recursive_quick_sort(arr, 0, arr.len() - 1);
}
\end{lstlisting}
\FloatBarrier

\newpage

\begin{lstlisting}[language=Rust, caption={Versão correta}]
fn _recursive_quick_sort<T: PartialOrd + Copy>(
  arr: &mut [T],
  lower_bound: usize,
  upper_bound: usize,
) {
  if lower_bound >= upper_bound {
    return;
  }
  let pivot_index = partition(arr, lower_bound, upper_bound);
  if pivot_index > 0 {
    _recursive_quick_sort(arr, lower_bound, pivot_index - 1);
  }
  _recursive_quick_sort(arr, pivot_index + 1, upper_bound);
}
\end{lstlisting}
\FloatBarrier

\subsubsection{Função auxiliar}

Note que ambas as versões do \textit{quick sort} utilizam a função auxiliar \texttt{partition}. Essa função seleciona o último elemento como pivô e rearranja a lista, de modo que todos os elementos menores que o pivó estejam na esquerda e os maiores ou iguais ao pivô na direita. Ao final, ela retorna o índice do pivô após a partição.
Os parâmetros \texttt{lower\_bound}, \texttt{upper\_bound} são os índices do elemento inicial e final do slice a ser particionado.

\begin{lstlisting}[language=Rust]
fn partition<T: PartialOrd + Copy>(arr: &mut [T], lower_bound: usize, upper_bound: usize) -> usize {
  let pivot = arr[upper_bound];
  let mut left_item = lower_bound as isize - 1;
  for right_item in lower_bound..upper_bound {
    if arr[right_item] < pivot {
      left_item += 1;
      arr.swap(left_item as usize, right_item);
    }
  }
  let new_pivot = (left_item + 1) as usize;
  arr.swap(new_pivot, upper_bound);
  new_pivot
}
\end{lstlisting}
\FloatBarrier

\subsection{Merge Sort}

\subsubsection{Iterativo}

Novamente, como o \textit{merge sort} naturalmente é um algoritmo recursivo, foi necessário simular a pilha de execução do computador. Primeiro, começamos com pequenos segmentos da lista (de tamanho 1) e iterativamente dobramos o tamanho dos segmentos a cada passo, mesclando-os.

\begin{lstlisting}[language=Rust]
pub fn iterative_merge_sort<T: PartialOrd + Copy>(arr: &mut [T]) {
  if arr.len() <= 1 {
    return;
  }
  let mut temp_arr = arr.to_vec();
  let mut segment_size = 1;
  let arr_len = arr.len();
  while segment_size < arr.len() {
    let mut start = 0;
    while start < arr_len {
      let mid = (start + segment_size).min(arr_len);
      let end = (start + 2 * segment_size).min(arr_len);
      merge(&mut temp_arr[start..end], &arr[start..mid], &arr[mid..end]);
      start += 2 * segment_size;
    }
    arr.copy_from_slice(&temp_arr);
    segment_size *= 2;
  }
}
\end{lstlisting}

\subsubsection{Recursivo}

A principal diferença em relação \href{algo:merge_sort_pseudo}{pseudo código}, é que a agora a função edita diretamente a lista, em vez de retorna-la como resultado.

\begin{lstlisting}[language=Rust]
pub fn recursive_merge_sort<T: PartialOrd + Copy>(arr: &mut [T]) {
  if arr.len() <= 1 {
    return;
  }
  let mid = arr.len() / 2;
  let mut left_arr = arr[..mid].to_vec();
  let mut right_arr = arr[mid..].to_vec();
  recursive_merge_sort(&mut left_arr);
  recursive_merge_sort(&mut right_arr);
  merge(arr, &left_arr, &right_arr);
}
\end{lstlisting}
\FloatBarrier

\subsubsection{Função auxiliar}

E a \textit{merge} também foi alterada de forma parecida, de forma que receba como argumento a lista resultado (\texttt{arr}).

\begin{lstlisting}[language=Rust]
fn merge<T: PartialOrd + Copy>(arr: &mut [T], left_arr: &[T], right_arr: &[T]) {
  let left_arr_len = left_arr.len();
  let right_arr_len = right_arr.len();
  let (mut i, mut l, mut r) = (0, 0, 0);
  while l < left_arr_len && r < right_arr_len {
    if left_arr[l] < right_arr[r] {
      arr[i] = left_arr[l];
      l += 1;
    } else {
      arr[i] = right_arr[r];
      r += 1;
    }
    i += 1;
  }
  while l < left_arr_len {
    arr[i] = left_arr[l];
    i += 1;
    l += 1;
  }
  while r < right_arr_len {
    arr[i] = right_arr[r];
    i += 1;
    r += 1;
  }
}
\end{lstlisting}

\subsection{Organização do projeto e corretude do algoritmo}

Para testar os algoritmos o projeto foi subdivido em duas partes, a que testa a performance e a que estipula a corretude, para tal foi usado a separação padrão do \texttt{cargo} entre o módulo principal e o módulo de testes. A organização dos arquivos ocorreu da seguinte forma:
\vspace{1em}
\dirtree{%
	.1 /.
	.2 out/.
	.3 entries.txt.
	.3 output.txt.
	.2 src/.
	.3 algorithms.rs.
	.3 lib.rs.
	.3 main.rs.
	.2 tests/.
	.3 test\_sorts.rs.
	.2 Cargo.lock.
	.2 Cargo.toml.
	.2 rustfmt.toml.
}
\FloatBarrier
\vspace{1em}
\noindent
No \texttt{Cargo.toml} A unica depêndencia usada foi \href{https://docs.rs/rand/latest/rand/}{rand}, para poder gerar números aleatórios de 0 a 100000
\begin{lstlisting}[caption={Trecho do \texttt{Cargo.toml}}]
  [dependencies]
  rand = "0.8.5"
\end{lstlisting}
\FloatBarrier
\noindent
Quando executado com \texttt{cargo test}, o \texttt{cargo} executa as funções de teste definidas em \texttt{test\_sorts.rs} para estipular a corretude de cada algoritmo. Os testes definidos foram:

\begin{lstlisting}[language=Rust, caption={Trecho de \texttt{test\_sorts.rs}}]
fn reverse_list_test(func: fn(&mut [i32])) {
  let mut arr = [5, 3, 2, 4, 1];
  func(&mut arr);
  assert_eq!(arr, [1, 2, 3, 4, 5], "reverse_list_test failed");
}

fn duplicates_list_test(func: fn(&mut [i32])) {
  let mut arr = [4, 2, 3, 2, 1, 4];
  func(&mut arr);
  assert_eq!(arr, [1, 2, 2, 3, 4, 4], "duplicates_list_test failed");
}

fn already_sorted_list_test(func: fn(&mut [i32])) {
  let mut arr = [1, 2, 3, 4, 5];
  func(&mut arr);
  assert_eq!(arr, [1, 2, 3, 4, 5], "already_sorted_list_test failed");
}

fn singleton_list_test(func: fn(&mut [i32])) {
  let mut arr = [42];
  func(&mut arr);
  assert_eq!(arr, [42], "singleton_list_test faild");
}

fn empty_list_test(func: fn(&mut [i32])) {
  let mut arr: [i32; 0] = [];
  func(&mut arr);
  assert_eq!(arr, [], "empty_list_test failed");
}
\end{lstlisting}
\FloatBarrier
\noindent
Todos os testes unitários foram definidos como esse:

\begin{lstlisting}[language=Rust, caption={Trecho de \texttt{test\_sorts.rs}}]
#[test]
fn test_recursive_quick_sort() {
  reverse_list_test(recursive_quick_sort);
  duplicates_list_test(recursive_quick_sort);
  already_sorted_list_test(recursive_quick_sort);
  singleton_list_test(recursive_quick_sort);
  empty_list_test(recursive_quick_sort);
}
\end{lstlisting}
\FloatBarrier
\noindent
Executando o comando \texttt{cargo test}, podemos ver que todos os testes foram bem sucedidos:

\begin{lstlisting}
running 6 tests
test test_iterative_bubble_sort ... ok
test test_iterative_merge_sort ... ok
test test_recursive_bubble_sort ... ok
test test_iterative_quick_sort ... ok
test test_recursive_merge_sort ... ok
test test_recursive_quick_sort ... ok

test result: ok. 6 passed; 0 failed; 0 ignored; 0 measured; 0 filtered out; finished in 0.00s
\end{lstlisting}
\FloatBarrier
\noindent
Os testes de performance foram implementados em \texttt{main.rs}, onde criam e escrevem os arquivos \texttt{entries.txt} e \texttt{output.txt} com as listas geradas e os resultados de performance, respectivamente.

\begin{lstlisting}[language=Rust, caption={Trecho de \texttt{main.rs}}]
type SortFn<T> = fn(&mut [T]);

fn main() -> io::Result<()> {
  let sort_functions = Vec::from([
    ("ITE BUBBLE SORT", iterative_bubble_sort as SortFn<i32>),
    ("REC BUBBLE SORT", recursive_bubble_sort),
    ("REC QUICK SORT", recursive_quick_sort),
    ("ITE QUICK SORT", iterative_quick_sort),
    ("REC MERGE SORT", recursive_merge_sort),
    ("ITE MERGE SORT", iterative_merge_sort),
  ]);

  fs::create_dir_all("out")?;
  let mut entries_file = File::create(ENTRIES_FILENAME)?;
  let mut output_file = File::create(OUTPUT_FILENAME)?;

  let title = "Performance test for sort algorithms";
  writeln!(output_file, "{}\n{}\n", title, "=".repeat(title.len()))?;
  writeln!(
    output_file,
    "ITE - stands for iterative\nREC - stands for recursive\n"
  )?;

  for n in 1..=4 {
    writeln!(entries_file, "List with {} entries:\n", TEN.pow(n))?;
    run_entry(
      &sort_functions,
      TEN.pow(n),
      &mut entries_file,
      &mut output_file,
    )?;
    writeln!(entries_file)?;
  }

  Ok(())
}
\end{lstlisting}
\FloatBarrier
