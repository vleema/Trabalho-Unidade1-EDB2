\section{Função iterativa}
\label{sec:func-it}

Ainda na proposta de comparar dois algoritmos que tem a mesma finalidade, o caso em questão diz respeito à duas funções que verificam se o menor elemento dela se repete. Ambas são testadas em listas de tamanho igual a 100, 1000 e 10000. E a única restrição de entrada é que os elementos da lista sejam números entre 1 e 100. Para isso, foram implementados em C++ dois algoritmos que realizam o procedimento descrito, e para cada tamanho de lista foram realizados testes com cinco listas diferentes. \\
Façamos a análise das diferentes implementações do mesmo algoritmo do ponto de vista do pior, melhor e caso médio.

\subsection{Algoritmo idadeRep}

O algoritmo \textbf{idadeRep} foi implementado da seguinte forma

\begin{lstlisting}[language=C++]
bool idadeRep(vector<int> &idade) {
  int tam = idade.size();
  int indexMenor = 0;
  for (int i = 1; i < tam; i++) {
    if (idade[i] < idade[indexMenor]) {
      indexMenor = i;
    }
  }
  for (int i = 0; i < tam; i++) {
    if (idade[i] == idade[indexMenor] && i != indexMenor) {
      return true;
    }
  }
  return false;
}
\end{lstlisting}

\subsection{Análise}

Note que o primeiro loop é inevitável independente do quão favorável seja a caso, e percorrerá a lista por inteira. Apenas no segundo loop que operações podem ser evitadas a depender da lista, pois o retorno de um valor encerra o programa. Além disso, perceba que a implementação acima pode ser visualizada da seguinte forma

\begin{align*}
  T(n) & = n + n\\ 
       & = 2n \\
       & = O(n)
\end{align*}

Sendo assim, temos que $O(n)$, $\Omega(n)$ e $\Theta(n)$.

\subsection{Resultados dos testes}


\begin{table}[h!]
	\centering
	\caption{Tabela de resultados de idadeRep}
	\label{tab:perf_result}
	\begin{tabular}{lrrrr}
		\toprule
		Lista   & n = 100   & n = 1000  & n = 10000 \\
		\midrule
		L1      & 431 ps    & 611 ps    & 3.837 ns  \\
		L2      & 200 ps    & 521 ps    & 3.236 ns  \\
		L3      & 251 ps    & 500 ps    & 3.186 ns  \\
		L4      & 160 ps    & 491 ps    & 3.086 ns  \\
		L5      & 130 ps    & 531 ps    & 3.346 ns  \\
		\midrule
		Médias  & 234.4 ps  & 530.8 ps  & 3.3382 ns \\
		\bottomrule
	\end{tabular}
\end{table}


