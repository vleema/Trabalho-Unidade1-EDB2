\section{Função Recursiva}
\label{sec:func-rec}

Novamente, iremos comparar dois algoritmos que tem a mesmo objetivo, nesse caso, as duas funções buscam um elemento X em uma lista. Ambas são testadas em listas de tamanho igual a 100, 1000 e 10000, sendo submetidas a cinco testes em cada uma. Para isso, foram implementados em C++ dois algoritmos que realizam o procedimento descrito. Sendo assim, façamos a análise das diferentes implementações do mesmo algoritmo do ponto de vista do pior, melhor e caso médio.

\subsection{Algoritmo buscaBinaria}

O algoritmo \textbf{buscaBinaria} foi implementado da seguinte forma

\begin{lstlisting}[language=C++]
int buscaBinaria(vector<int> &lista, int tamanho, int x) {
    int esq = 0; 
    int dir = tamanho - 1;

    while (esq < dir) {
        int m = (esq + dir) / 2; 
        if (x > lista[m]) {
            esq = m + 1;
        } else {
            dir = m;
        }
    }

    if (lista[esq] == x) {
        return esq; 
    }

    return -1;
}
\end{lstlisting}

\subsection{Análise do buscaBinaria}

Note que o loop tem a condição de parada (esq < dir), o que significa que ele só irá terminar quando sobrar somente um valor possível na lista. Portanto, independente do quão favorável seja o caso, ele não irá verificar se lista[m] é o valor que está tentando encontrar, assim executado o loop aproximadamente $\log_ 2(\text{tamanho})$ vezes em todos os casos. Perceba que o algoritmo necessita de melhorias para funcionar de forma que a posição do número buscado seja influente na complexidade do código. Portanto, o loop se comportará da seguinte maneira:

\begin{align*}
        n/2^k = 1 
       \Longrightarrow n = 2^k
       \Longrightarrow k = \log_2 n
\end{align*}

Em suma teremos o seguinte

\begin{itemize}
  \item \textbf{Casos:} Dado que o loop inevitávelmente rodará $\log_2 ({tamanho})$ vezes independente do cenário da lista, todos os casos terão a mesma complexidade.
 
\end{itemize}

\subsection{Resultados do buscaBinaria}

Portanto, o algoritmo \textbf{buscaBinaria} teve a seguinte performace nos testes

\begin{table}[h!]
	\centering
	\caption{Tabela de resultados de buscaBinaria}
	\label{tab:busc_bin_result}
	\begin{tabular}{lrrrr}
		\toprule
		Lista   & n = 100   & n = 1000  & n = 10000 \\
		\midrule
		L1      & 190 ps    & 51 ps    & 141 ps  \\
		L2      & 80 ps    & 70 ps    & 110 ps  \\
		L3      & 80 ps    & 70 ps    & 101 ps  \\
		L4      & 50 ps    & 80 ps    & 100 ps  \\
		L5      & 50 ps    & 60 ps    & 100 ps  \\
		\midrule
		Médias  & 90 ps  & 66.2 ps  & 110.4 ps \\
		\bottomrule
	\end{tabular}
\end{table}

É perceptivel que a Lista 1 com 100 números foi um outlier em nossa tabela, alterando o valor da média drasticamente tornando-a atípica. Entretanto podemos ver como nos outros casos o tempo é crescente levando em conta o tamanho da lista.

\subsection{Algoritmo bBinRec}

Já o algoritmo \textbf{bBinRec} foi implementado da seguinte maneira 

\begin{lstlisting}[language=C++]
int bBinRec(vector<int> &lista,int esq, int dir, int x ){
    if (esq>dir) {
        return -1;
    }
    int m = (esq + dir) / 2;
    if (lista[m] == x) {
        return m;
    }
    if (lista[m] > x){
        return bBinRec(lista, esq, m - 1, x);
    }else{
        return bBinRec(lista, m + 1, dir, x);
    }
}
\end{lstlisting}

\subsection{Análise do bBinRec}

É possível reparar que a função é diferente da \textbf{buscaBinaria} pois chama a si mesma em cada passo. Cada chamada reduz o intervalo de busca, mas em vez de usar um loop, ela continua chamando a função com novos valores. Outra diferença é que a função \textbf{bBinRec} verifica \texttt{lista[m] == x} em cada chamada, permitindo que a função retorne imediatamente se o elemento for encontrado. Portanto, isso mudará a complexidade, tendo casos destintos, diferente do algoritmo anterior.
Em suma teremos o seguinte:

\begin{itemize}
	\item \textbf{Melhor Caso:} $O(1)$ — o elemento está no meio na primeira chamada.
	\item \textbf{Caso Médio:} $O(\log_2 n)$ — o elemento está em uma posição aleatória e requer, em média, $\log_2 n$ divisões.
	\item \textbf{Pior Caso:} $O(\log_2 n)$ — o elemento não está presente ou está no final da lista, e o algoritmo precisa dividir o intervalo até se esgotar.
\end{itemize}

\subsection{Resultados do bBinRec}

A versão recursiva é ligeiramente mais rápida na prática, pois pode interromper o processo assim que encontra o elemento, sem precisar fazer verificações adicionais no intervalo.
Portanto, o algoritmo \textbf{bBinRec} teve a seguinte performace nos testes:

\begin{table}[h!]
	\centering
	\caption{Tabela de resultados de bBinRec}
	\label{tab:bBin_rec_result}
	\begin{tabular}{lrrrr}
		\toprule
		Lista   & n = 100   & n = 1000  & n = 10000 \\
		\midrule
		L1      & 80 ps    & 60 ps    & 70 ps  \\
		L2      & 40 ps    & 51 ps    & 71 ps  \\
		L3      & 60 ps    & 50 ps    & 50 ps  \\
		L4      & 50 ps    & 50 ps    & 60 ps  \\
		L5      & 50 ps    & 50 ps    & 60 ps  \\
		\midrule
		Médias  & 56 ps  & 52.2 ps  & 62.2 ps \\
		\bottomrule
	\end{tabular}
\end{table}

Novamente, perceptivel que a Lista 1 com 100 números foi um outlier em nossa tabela, alterando o valor da média drasticamente tornando-a atípica. Entretanto podemos ver como nos outros casos o tempo é crescente levando em conta o tamanho da lista.

\subsection{Comparação dos algoritmos}

É possível perceber que o \textbf{bBinRec} performa melhor em todos os testes. De forma mais específica

\begin{itemize}
  \item \textbf{n = 100:} A média de \textbf{buscaBinaria} é de 90 ps, enquanto a de \textbf{bBinRec} é de 56 ps.

  \item \textbf{n = 10000:} Já nesse, caso \textbf{buscaBinaria} apresenta média de 66,2 ps, enquanto \textbf{bBinRec} apresenta 52,2 ps.

  \item \textbf{n = 100000:} Por fim, \textbf{buscaBinaria} tem 110.4 ps de média. Já \textbf{bBinRec}, 62,2 ps.
\end{itemize}

Assim, ao comparar os resultados dos algoritmos lado a lado, torna-se evidente que a implementação de \textbf{bBinRec} se destaca.
